\newif\ifdraft\drafttrue  % set true to show comments
\newif\iflastminute\lastminutetrue  % for things to check at the end
\newif\iflater\laterfalse  % for things to think about after submission

\documentclass[acmsmall,review,anonymous]{acmart}
\settopmatter{printfolios=true,printccs=false,printacmref=false}
% % For double-blind review submission, w/ CCS and ACM Reference
% \documentclass[acmsmall,review,anonymous]{acmart}\settopmatter{printfolios=true}
% % For single-blind review submission, w/o CCS and ACM Reference (max
% submission space)
% \documentclass[acmsmall,review]{acmart}\settopmatter{printfolios=true,printccs=false,printacmref=false}
% % For single-blind review submission, w/ CCS and ACM Reference
% \documentclass[acmsmall,review]{acmart}\settopmatter{printfolios=true} % For
% final camera-ready submission, w/ required CCS and ACM Reference
% \documentclass[acmsmall]{acmart}\settopmatter{}


% % Journal information % Supplied to authors by publisher for camera-ready
% submission; % use defaults for review submission.
\acmJournal{PACMPL}
\acmVolume{1}
\acmNumber{ICFP} % CONF = POPL or ICFP or OOPSLA
\acmArticle{1}
\acmYear{2018}
\acmMonth{1}
\acmDOI{} % \acmDOI{10.1145/nnnnnnn.nnnnnnn}
\startPage{1}

% % Copyright information % Supplied to authors (based on authors' rights
% management selection; % see authors.acm.org) by publisher for camera-ready
% submission; % use 'none' for review submission.
\setcopyright{none}
\usepackage{algorithm, amsmath, amssymb, verbatim, enumerate, graphicx,
  centernot, tikz, array, mathtools, bussproofs, stmaryrd, enumitem,
  stackengine, subcaption}
\captionsetup{compatibility=false}
\usepackage{relsize}
\usepackage{listings}
\usepackage{proof}
\usepackage{xspace}
\usepackage[noend]{algpseudocode}
\usepackage[capitalize]{cleveref}

\lstset{ language=Caml, basicstyle=\linespread{0.8}\upshape\sffamily,
keywordstyle=\upshape\sffamily\color{dkblue}, keepspaces=true,
framexleftmargin=1ex, framexrightmargin=1ex, showstringspaces=true,
commentstyle=\itshape\rmfamily,
emph={synth,collapse,perm,squash,normalize,using,ins,del,lens,let,get,put,rquot,lquot},
emphstyle=\upshape\sffamily\color{dkblue}, 
columns=fullflexible,
escapechar=\#,
mathescape, 
xleftmargin=1.5em,
% BCP: I find this distracting:
% stringstyle=\sffamily\color{dkred},
}
\makeatletter
     \let\lst@oldvisiblespace\lst@visiblespace
     \def\lst@visiblespace{\,\lst@oldvisiblespace\,}
\makeatother

% %%%% Macros Colors
\definecolor{dkblue}{rgb}{0,0.1,0.7}
\definecolor{dkgreen}{rgb}{0,0.6,0}
\definecolor{dkred}{rgb}{0.6,0,0}
\definecolor{dkpurple}{rgb}{0.7,0,0.4}
\definecolor{olive}{rgb}{0.4, 0.4, 0.0}
\definecolor{teal}{rgb}{0.0,0.5,0.5}
\definecolor{orange}{rgb}{0.9,0.6,0.2}
\definecolor{lightyellow}{RGB}{255, 255, 179}
\definecolor{lightgreen}{RGB}{170, 255, 220}
\definecolor{teal}{RGB}{141,211,199}
\definecolor{darkbrown}{RGB}{121,37,0}

\newcommand{\FINISH}[3]{\ifdraft\textcolor{#1}{[#2: #3]}\fi}
\newcommand{\bcp}[1]{\FINISH{dkred}{B}{#1}}
\newcommand{\BCP}[1]{\FINISH{dkred}{B}{\bf #1}}
\newcommand{\afm}[1]{\FINISH{dkgreen}{A}{#1}}
\newcommand{\dpw}[1]{\FINISH{dkblue}{D}{#1}} % Toronto Maple Leafs Blue :-)
\newcommand{\saz}[1]{\FINISH{orange}{SZ}{#1}}
\newcommand{\SAZ}[1]{\FINISH{orange}{SZ}{#1}}
\newcommand{\ksf}[1]{\FINISH{teal}{K}{#1}}
\newcommand{\sam}[1]{\FINISH{dkpurple}{SM}{#1}}

% For Inference Rules
\newcommand{\Rule}[2]{\infer{#2}{#1}}
\newcommand{\RuleSide}[3]{\infer[#3]{#2}{#1}}
% \newcommand{\Axiom}[1]{\Rule{}{#1}}

\newcommand{\wf}[1]{\ensuremath{#1\;\mathsf{wf}}}

% FOR Regular Expression names
\newcommand{\re}[1]{\ensuremath{\mathtt{#1}}}
\newcommand{\codefont}[1]{\ensuremath{\mathsf{#1}}}
\newcommand{\kw}[1]{\textcolor{dkblue}{\ensuremath{\mathsf{#1}}}}
\newcommand{\collapse}[2]{\ensuremath{\kw{collapse} \; #1 \mapsto #2}}
\newcommand{\squash}[3]{\ensuremath{\kw{squash} \; #1 \rightarrow #2\; \kw{using} \; #3}}
\newcommand{\perm}[2]{\ensuremath{\kw{perm}(#1)\; \kw{with}\; #2}}
\newcommand{\normalize}[3]{\ensuremath{\kw{normalize}(#1, #2, #3)}}
\newcommand{\eqrel}[1]{\ensuremath{\equiv_{#1}}}
\newcommand{\sep}{\ensuremath{\ | \ }}
\newcommand{\canonize}{\ensuremath{\kw{canonize}}}
\newcommand{\bibtex}{\textsc{Bib}\TeX{}}
\newcommand{\get}{\ensuremath{\kw{get}}}
\newcommand{\semicolon}{\ensuremath{\; ; \;}}
\newcommand{\lput}{\ensuremath{\kw{put}}}
\newcommand{\create}{\ensuremath{\kw{create}}}
\newcommand{\const}{\ensuremath{\kw{const}}}
\newcommand{\swap}{\ensuremath{\kw{swap}}}
\newcommand{\id}{\ensuremath{\kw{id}}}
\newcommand{\lquot}{\ensuremath{\kw{lquot}}}
\newcommand{\rquot}{\ensuremath{\kw{rquot}}}
\newcommand{\lift}{\ensuremath{\kw{lift}}}
\newcommand{\bsep}{\ \ \sep{} \ \ }

\newcommand{\Name}{Optometrist\xspace}

\newcommand{\QRESize}{\textbf{QS}}
\newcommand{\canonizeAndSpecSize}{\textbf{BS}}
\newcommand{\LensAndSpecSize}{\textbf{NS}}

%\newcommand{\QOpt}{Optician_Q}
\newcommand{\QOpt}{QRE-enhanced Optician}
\newcommand{\OpticianRuntime}{\textbf{Optician}}
\newcommand{\QREOptician}{\textbf{Optician\textsubscript{Q}}}
\newcommand{\SystemOnOptician}{\textbf{QO}}
\newcommand{\SystemOnBenchmarks}{\textbf{QQ}}
\newcommand{\cd}[1]{\lstinline[backgroundcolor=\color{white}]$#1$}


% %%%%%%%%%%%%%%%%%%%%%%%%%%%%%%%%%%

\begin{document}
\title{Synthesizing Quotient Lenses: Artifact Evaluation}
\maketitle

\section{Setup Environment}
\subsection{Virtual Machine Environment (recommended)}
To make validation easier, we provide the Artifact Evaluation Committee with a
virtual machine that has all components installed.  Setup for the virtual
machine environment detailed below:
\begin{enumerate}
\item Install a Virtual Machine Manager (we tested with VirtualBox)
\item Download our virtual machine image (QuotientLensSynthesis.ova in TODO)
\item Load our virtual machine from your virtual machine manager
\end{enumerate}

The user for the virtual machine is solomon, with password quotientsynthesis.

\subsection{Custom Environment}
If you want to install on a custom environment, certain programs and libraries
must be installed.  We provide the commands for installation in Lubuntu 17.10
in parenthesis next to the program or library name.

\begin{enumerate}
\item Install opam (sudo apt install opam; opam init; eval `opam config env`;
  opam update)
\item Switch opam to OCaml 4.06.0 (opam switch 4.06.0; eval `opam config env`)
\item Install necessary system packages (opam depext conf-m4.1)
\item Install OCaml's Core version 0.10.0 (opam install core.v0.10.0)
\item Install OCaml's ppx\_deriving (opam install ppx\_deriving)
  
\item Install python-pip (sudo apt install python-pip)
\item Install Python's EasyProcess (pip install EasyProcess)
\item Install Python's matplotlib (pip install matplotlib)
\item Install Python's tk (sudo apt install python-tk)
\item Install \LaTeX \; (sudo apt install texlive-full)

\item Install the codebase (git clone
https://github.com/SolomonAduolMaina/boomerang.git)
 
\end{enumerate}

\section{Tool Validation}
To validate, navigate to the directory the codebase is installed in
(/home/boomerang in the virtual machine), switch to the eval-pt2
branch (git checkout eval-pt2), and then run the following commands:
\begin{enumerate}
\item make generate-data
\item make generate-graphs
\end{enumerate}

After these commands are run, the figures will be present at the following
locations: 
\begin{itemize}
\item Figure 10 will be present at \$/generated-graphs/times.eps
\item Figure 11(a) will be present at \$/generated-graphs/times\textunderscore
new.eps
\item Figure 11(b) will be present at \$/generated-graphs/times\textunderscore
opt.eps
\end{itemize}

The command ``make generate-data'' will take about 15 minutes to complete, while
the command ``make generate-graphs'' will take about 5 seconds to complete.

\subsection{Directory Information}
The command make generate-data creates five csv files in the
generated\textunderscore data folder:
\begin{enumerate}
  \item
  The file
\$/generated\textunderscore data/new\textunderscore comparison\textunderscore
data.csv contains the running times for synthesizing lenses from the
examples/synthexamples/new\textunderscore specs folder. This folder contains the
benchmarks used for the Optician tool where the lenses
synthesized involved non-trivial canonization i.e. the canonization function was
not the identity function. This file is used to create Figure 11 (b) when the
command ``make generate-graphs'' is invoked.
\item
The file
\$/generated\textunderscore data/optician\textunderscore
comparison\textunderscore data.csv contains the running times of synthesizing
lenses from the examples/synthexamples/new\textunderscore
optician\textunderscore specs folder using the Optician tool. These are the
benchmarks used in the Optician tool, with the lenses being bijective. This file
is used to create Figure 11 (a) when the command ``make generate-graphs'' is
invoked.
\item
The file
\$/generated\textunderscore data/new\textunderscore optician\textunderscore
comparison\textunderscore data.csv contains the running times of synthesizing
lenses from the examples/synthexamples/new\textunderscore
optician\textunderscore specs folder using our new tool implement in Boomerang.
These are the benchmarks used in the Optician tool with the lenses synthesized
being quotient lenses. This file is used to create Figure 11 (a) when the
command ``make generate-graphs'' is invoked.
\item
The file \$/generated\textunderscore data/size \textunderscore data.csv contains
the size of the lenses synthesized from the 
examples/synthexamples/new\textunderscore specs folder.This file is used to
create Figure 10 when the command ``make generate-graphs'' is invoked.
\end{enumerate}

In addition to this, the \$/generated\textunderscore data folder contains the
pre-generated file oo\textunderscore data.csv file which which contains various
statistics on the lenses synthesized from the Optician benchmarks. This file
is also used to create Figure 10 when the command ``make generate-graphs is
invoked. The file was pre-generated since generating it takes a very long time.

The file QuickStart.src provides a quick introduction to Boomerang programming.

Our new tool integrates the Optician lens synthesis tool \cite{optician} into
the lens programming language \cite{boomerang} and provides an implementation of
quotient regular expressions (QREs) and QRE lenses. Consequently, the file
structure of our tool inherits the file structure of each of these two systems.

We encourage the evaluation committee to consult\\
https://github.com/Miltnoid/BijectiveLensSynthArtifactEvaluation and
\\http://www.seas.upenn.edu/~harmony/ for more information on Optician and
Boomerang.
\bibliographystyle{plain}
\bibliography{local}

\end{document}
